\section{CUD教育について}
本章では,CUD教育の内容と現状の課題について説明する.

\subsection{概要}
一般的に,色覚異常に関する講習では,主に色覚異常の見え方や色覚異常者にも伝わりやすいデザインであるCUDについての説明を行う.
CUD教育では,学習者が色覚異常に関する知識を得るだけでなく,資料等を作成する際にCUDに配慮した工夫を取り入れることができるようになることが求められる.
そのため,講義形式での説明に加え,学習者にCUDに配慮したデザインを体験してもらう場面が存在する.

大阪医療福祉専門学校でのCUDに関する授業では,色選びの体験ワークを行った\cite{jcolor}.
まず,学習者にゴミ分別に関する図の配色を考えさせた.
そして,色覚異常者の見え方を体験できるフィルター式眼鏡を掛けて,選択した色が色覚異常者でも区別できる配色となっているか確認させた.
このように,実際にCUDに配慮した図の作成を体験させる授業が行われた.

また,東京女子大学での講義では,色覚異常に関する説明の後にCUDに配慮した資料作成を実践する機会を設けていた\cite{sugamiya}.
この講義では,まず,東京都が作成したガイドライン等を参考に,色覚異常や色覚異常者にも伝わりやすい資料作りとして色のバリアフリーについての説明を行った.
その上で,受講者に色覚異常者に配慮した発表資料を作成させる等,CUDを取り入れた資料作成を実践する機会を設けている.

これらのように,学習者に資料等を作成する際にCUDに配慮した工夫を取り入れることができるようになることを目的とした教育が行われている.

\subsection{CUD教育における課題}
受講者に資料作成を実践させた授業に関する先行研究によると,講習を受けた学生の20%がCUDに配慮していない資料を作成した\cite{sugamiya}.
先行研究では,初めて色覚異常の知識に触れる学生にとって複数の別色覚にいきなり配慮することは難しく,CUDを直ぐに理解して取り入れることは困難であるためと考察されている.
そのため,資料作成を実践する前段階として,CUDに配慮した工夫を一つ一つ実践する機会が必要だと考える.

学習者は別色覚での見え方を確認することでCUDをより理解できると考えられる.
別色覚での見え方の確認は,シミュレータを利用することで可能となるが,現状では資料作成ツールとは別媒体のシミュレータを通す必要がある.
また,評価者にとって学習者の作成物のCUD適合度を即座に判断することは難しく,評価や改善案を即座に提示することは困難である.
そのため,学習の際は,別色覚での見え方を即座に確認でき,自動で評価・改善案の提示を行う環境が望ましい.






