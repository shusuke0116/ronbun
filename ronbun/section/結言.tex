\section{結言}
遺伝子の異常により通常と色の見え方が異なる色覚を持つ色覚異常者は,男性で5%,女性で0.2%存在する.
色覚の多様性に配慮し,より多くの人に伝わりやすいデザインとしてカラーユニバーサルデザイン(CUD)がある.

学生や社会人になると,ゼミや業務等で資料を作成する機会が増加する.
そのため,CUDの講習では資料作成を実践し評価する機会が設けられている.
しかし,講習を受けた学生の20%がCUDに配慮していない資料を作成した.

本研究では,CUDに配慮した工夫を一つ一つ実践する機会が必要だと考え,これらを行える学習ツールを開発した.
開発した学習ツールを学習者が利用した結果,配色に関する項目で,先行研究に比べてCUDに配慮した資料が増加した.
このことから,開発した学習ツールを利用することで,CUDに配慮した資料を作成できるという結果が得られた.
一方,評価項目の一つである装飾の観点でCUDに配慮された資料は70%であり,他項目に比べて工夫が行われていないため,学習の流れや表示項目等の改善が望まれる.