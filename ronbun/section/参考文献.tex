\begin{thebibliography}{99}

\bibitem{okabe}岡部正隆・伊藤啓・橋本知子: ``ユニバーサルデザインにおける色覚バリアフリーへの提言'', \url{https://www.nig.ac.jp/color/handout1.pdf}, 2024/1/9参照

\bibitem{sugamiya} 菅宮恵子: ``色覚異常を考慮した教材資料作成実習の実践報告とその評価'', 教職・学芸員課程研究,2号(2020),p.14-23, 2024/1/9参照

\bibitem{tokyo} 東京都福祉保健局生活福祉部地域福祉推進課: ``東京都カラーユニバーサルデザインガイドライン'', \url{https://www.fukushi.metro.tokyo.lg.jp/kiban/machizukuri/kanren/color.files/colorudguideline.pdf},2024/1/9参照

\bibitem{osaka} 府民文化部府政情報室広報広聴課: ``色覚障がいのある人に配慮した色使いのガイドライン'', \url{https://www.pref.osaka.lg.jp/koho/shikikaku/guide1.htmlf},2024/1/9参照

\bibitem{jcolor} 一般社団法人日本カラーコーディネーター協会: ``大阪医療福祉専門学校様 特別授業「CUD」のご報告'', \url{https://www.j-color.or.jp/2023/03/15/blog-entry-884/},2024/1/9参照

\bibitem{simulator}浅田一憲: ``色のシミュレータ'', \url{https://asada.website/cvsimulator/j/index.html},2024/1/9参照




\end{thebibliography}
