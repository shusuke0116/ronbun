\section{緒言}

%背景1
遺伝子の異常により通常と色の見え方が異なる色覚を持つ色覚異常者は,男性で5%,女性で0.2%存在する\cite{okabe}.
色覚異常には,主に赤を感じづらいP型色覚,緑を感じづらいD型色覚,青を感じづらいT型色覚がある.
色覚の多様性に配慮し,より多くの人に伝わりやすいデザインとしてカラーユニバーサルデザイン(CUD)がある.

%背景2
学生や社会人になると,ゼミや業務等で資料を作成する機会が増加する.
そのため,CUDの講習では資料作成を実践し評価する機会が設けられている.
しかし,講習を受けた学生の20%がCUDに配慮していない資料を作成した\cite{sugamiya}.
先行研究では,講習内容を即座に理解することが困難なためと考察されている.
そのため,CUDに配慮した工夫を一つ一つ実践する機会が必要だと考える.

%問題点
学習者は別色覚での見え方を確認することでCUDをより理解できると考えられるが,現状では別媒体のシミュレータを使う必要がある.
また,評価者にとって学習者の作成物のCUD適合度を即座に判断することが難しく,評価や改善案を即座に提示することは困難である.
そのため,学習の際は,別色覚での見え方を即座に確認でき,自動で評価・改善案の提示を行う環境が望ましい.

%目的
そこで本研究では,これらの環境を持ち,CUDに配慮した工夫を項目ごとに実践できる学習ツールの開発を行う.
そして,開発した学習ツールを学習者が利用することでCUDに配慮した資料を作成できるか検証することを目的とする.

