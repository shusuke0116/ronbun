\section{検証}
この章では,開発した学習ツールを用いて行った検証について説明する.

\subsection{実施内容}
開発した学習ツールを用いて,学生10名を対象に検証した.
対象者には,東京都が作成したCUDのガイドラインを参考に色覚異常とCUDについて事前に説明した\cite{tokyo}.
そして,対象者には学習ツールを利用した後に,Microsoft PowerPointを用いてスライドを作成してもらった.
そのスライドをCUDに配慮されているか項目別に評価した.

\subsection{評価項目}
評価項目は,適切なフォント,色の組み合わせ,強調目的の色,ハッチング,装飾の5項目である.
これは先行研究での評価項目に,装飾に関する項目を加えた.
配色に関する項目については,医学及びメディアデザイン学の博士号を持つ作者が開発した色のシミュレータを用い,通常の色覚に加え,P型色覚,D型色覚,T型色覚での見え方を確認し評価した\cite{simulator}.

「適切なフォント」では,使用しているフォントがCUDに配慮されているか評価した.
線の一部が細くなっているフォントは色面積が小さくなり,色による判別がつきづらくなるため,色による文字の強調を行う際には適さないフォントである.
そのため,線の幅が一定なフォントが使用されているかという観点で評価した.

「色の組み合わせ」では,2色を組み合わせて使用する場合に,どの色覚であっても区別がつきやすい配色となっているか評価した.

「強調目的の色」では,文字の一部の色を別の色にすることで強調する場合に,どの色覚でも強調していることが伝わりやすい配色となっているか評価した.
通常の色覚では黒と明確に区別できる色でも,別の色覚では黒と区別がつきづらい色に見える場合がある.
複数の色覚での見え方を確認し,どの色覚でも明確に区別がつき,強調が伝わる配色となっているか評価した.

「ハッチング」では,折れ線グラフを作成する際に,それぞれの線を色以外の情報で区別できるようになっているか評価した.
折れ線グラフを作成する際は,実線と破線を組み合わせ,マーカーの図形を線ごとに別の種類にすることで,色に頼らずに線を区別できるため,色覚によらず伝わりやすいグラフとなる.
これらの工夫が取り入れられているか評価した.

「装飾」では,色以外で文字の強調が伝わる資料となっているか評価した.
文字を強調する際は,配色に加え,アンダーラインを引くことや,強調する文字を太くする等を行うことで,色以外の情報で文字の強調を伝えることができる.
これらの工夫が取り入れられているか評価した.
