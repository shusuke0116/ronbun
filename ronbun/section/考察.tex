\section{考察}
配色に関する項目で,先行研究に比べてCUDに配慮した資料が増加している.
開発した学習ツールを利用することで,CUDに配慮した資料を作成できるという結果が得られた.
そのため,学習者がCUDに配慮した工夫を項目ごとに実践することで,CUDに配慮した資料を作成できると考えられる.
また,配色に関する項目でCUDに配慮した資料が増加した要因として,本学習ツールで別色覚での見え方や評価を即座に確認できるようになったことで,様々な配色を試すことが容易であったことが考えられる.

一方,評価項目の一つである装飾の観点では,CUDに配慮された資料は70%であった.
本学習ツールでは,配色に関する項目の後に装飾を学習する.
複数の色覚で区別がつきやすい配色を行った後に装飾に関する工夫を実践する流れであることや,配色と異なり複数回試すことがない項目であることから,装飾の重要性が伝わりにくくなっていたと考えられる.
これらから,学習の流れや表示項目を改善することが望ましい.